\chapter{Question 5 - Network Forensics}

\section{Preparation}
Prep goes here

\section{Given Information}
All teams were given a file: network-forensics.zip\\
The file contained a network capture file (PCAP).

\section{Question 5.1}
\textbf{a) What is the CVE of the vulnerability used to exploit the mail server?
\\b) What was the operating system that was targeted?
\\\\
Example answer format: [CVE-1234-1234] [msdos]}
\subsection{Approach}
Approach details go here
\subsection{Retrospective Approach}
Retrospective approach details go here

\section{Question 5.2}
\textbf{The first payload contains shellcode that invokes system calls i.e.
"int80". The first system call is "socket". What are the next two system calls
(remember the target OS)?
\\\\
Example answer format: [syscall1] [syscall2]}
\subsection{Approach}
Approach details go here
\subsection{Retrospective Approach}
Retrospective approach details go here

\section{Question 5.3}
\textbf{Deobfuscate the second payload to reveal the triple DES key. The key is
the flag}
\subsection{Approach}
Approach details go here
\subsection{Retrospective Approach}
Retrospective approach details go here

\section{Question 5.4}
\textbf{Once the third payload is decrypted with the above key, it drops out a
binary. Deobfuscate the key hidden in the binary. The key is the flag}
\subsection{Approach}
Approach details go here
\subsection{Retrospective Approach}
Retrospective approach details go here

\section{Question 5.5}
\textbf{Decode the communication and retrieve the flag}
\subsection{Approach}
Approach details go here
\subsection{Retrospective Approach}
Retrospective approach details go here
