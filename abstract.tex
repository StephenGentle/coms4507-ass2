\chapter{Abstract}

This report is a description and analysis of the questions and tasks that the 
University of Queensland 1 (UQ1) team answered in the recent Cyber Security
Challenge Australia (CySCA) competition, which was run jointly by Telstra and
the Defence Signals Directorate (DSD).

The team attempted questions from many sections of the competition, and
captured many of the flags. Overall, the team achieved twelfth place out of
forty-three teams.

In Question 1, the team captured two of the four flags, exploiting an
information disclosure vulnerability where a secret file was able to be found
because the server had directory listing enabled. The second flag involved
exploiting an SQL injection vulnerability in a sign-up script, which allowed
a user to be created that was already approved (as opposed to how it was
intended to work - requiring an administrator to approve all new accounts).

In Question 2, a corporate network was penetration tested. The team succeeded
in gaining access to a machine on the corporate network by creating a
malicious web site that contained a Java applet that exploited a vulnerability
in the outdated Java version installed. This enabled a saved password to be
stolen from the user's browser profile. Unfortunately, a privilege escalation
method was not found, so the subsequent flags were not able to be captured.

The third question involved assessing application code to find vulnerabilities.
A client and server application, written in Python, was supplied to the group.
The applications used an imaginary protocol called the Very Secure Transfer
Protocol. The team had to first find a command injection vulnerability, which
was exploited to be able to download a file from the server which was outside
the transfer server's document root. After succeeding in this, a network
capture file, also found on the VSTP server, was downloaded, and a
cryptographic weakness in the protocol had to be found so the packets in the
capture could be decrypted. This turned out to be in the session key
generation algorithm, which was very badly designed and created a key with only
thirty bits of entropy, which is easily found with a brute-force algorithm.
Finding the key made decrypting the file transferred in the capture possible,
and the flag was found. The last flag required exploiting a buffer overflow
vulnerability in a custom Snort IDS plugin, and although the issue was
easily found, an exploit was not able to be formulated for it.

The fourth question... 

In Question 5, only the first question was answered by the team. Solutions to
the following parts were unfortunately not found.

The final question involved analysing a set of configuration files for an 
OpenVPN setup. Some of these were successfully answered, but this question was
left until too late in the competition to find answers to all of them. This
was unfortunate, because it would have helped push the team's ranking up. It
had been left until late in the competition because capturing flags seemed to
make a much bigger improvement to rankings than the questions.

Overall, the challenge was extremely enjoyable, and all the members of the
group leant a great deal about computer security and penetration testing.
